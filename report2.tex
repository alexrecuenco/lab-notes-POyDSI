% !TEX TS-program = pdflatexmk
\documentclass{./packages/optica-article}

\graphicspath{{images/}{practica2/images}}

\journal{opticajournal}


\usepackage{csvsimple}
\usepackage{siunitx}
\usepackage{physics}
\usepackage{booktabs}
\usepackage{tikz}
\usetikzlibrary{positioning}

\tikzset{>=stealth}

% cref instead of ref
\usepackage{cleveref}

\newcommand{\sinc}{\textrm{sinc}}
\newcommand\conv{\circledast}


% Set the article type
\articletype{Research Article}

% \usepackage{lineno}
% \linenumbers

\begin{document}

\title{El efecto Talbot}

\author{Adriana Mamani Lazarte\authormark{1} Alex G. Recuenco\authormark{1}, and Carlos España Castaño\authormark{1}}

\address{\authormark{1}Universidad Complutense de Madrid, Madrid, PC 28040, España}

\section{Introducción}

\section{Setup O COMO LO LLAMES}

\subsection{Teórico}

\subsection{Descripción del setup la práctica O COMO SE DIGA}

\section{Resultados experimentales}

\subsection{Observación del efecto Talbot}

\subsubsection{Representando una señal óptica periódica (longitud de onda $\lambda$, periodo $d$) como una serie de Fourier,
	demostrar analíticamente que en las distancias iguales a la mitad de la distancia de Talbot,
	$z=\flatfrac{ZT}{2}$, observamos el objeto original desplazado en la mitad del periodo $d$.
}
\subsubsection{Usando las ecs. (7)-(8) analizar la intensidad y la fase del campo difractado por un objeto periódico compuesto de líneas blancas (transparentes) y negras (opacas) alternadas de la misma anchura, a una distancia $z=\flatfrac{ZT}{4}$}
\subsubsection[]{\begin{enumerate}
		\item Justifique el procedimiento seguido para determinar la distancia de Talbot y el periodo de la red y exponga los resultados.
		\item ¿Las imágenes observadas siempre son periódicos? ¿La respuesta está de acuerdo con la teoría?
	\end{enumerate}}


\subsection{Observación del espectro de Fourier}
\subsubsection{¿Qué valores d1 y d2 son adecuados para la observación del espectro de Fourier?}
\subsubsection[]{\begin{enumerate}
		\item ¿El espectro de la red está de acuerdo con la teoría (véanse los resultados del problema 1.6.)?
		\item Justifique el procedimiento seguido para determinar el periodo de la red y exponga los resultados.
	\end{enumerate}}
\subsubsection{¿Es posible registrar el espectro de Fourier colocando un objeto detrás de la lente?}



\subsection{Filtrado óptico}
\subsubsection{¿Qué distancia focal tiene la lente más próxima a la camera? Justifique el procedimiento seguido para determinar el orden de las lentes.}
\subsubsection{Elegir un filtro entre los estudiados en la práctica que podría ser adecuado para la observación de objetos semi-transparentes. Justifique la respuesta.}
\subsubsection{¿Qué observamos en la salida del sistema si usamos como filtro una red periódica? Justifique la respuesta.}

\subsection{Formación de imágenes con luz parcialmente coherente}

\subsubsection{¿El rango de distancias donde observamos la imagen del objeto 2D es mayor en el caso iluminación coherente o parcialmente coherente? ¿Que illuminación es mejor para la observación de objetos 3D?}

\subsubsection{Para el sistema de filtrado usada en la práctica la ultima lente es la pupila de salida del sistema. Determinar la respuesta impulsional y la función de transferencia del sistema en el caso de iluminación (a) coherente; (b) incoherente.}
%%%%%%%%%% If using BibTeX:
\bibliography{bibliography}

\end{document}
