% !TEX TS-program = pdflatexmk
\documentclass{./packages/optica-article}

\graphicspath{{images/}{practica1/images}}

\journal{opticajournal}

\usepackage{csvsimple}
\usepackage{physics}
\usepackage{booktabs}
% cref instead of ref
\usepackage{cleveref}


% Set the article type
\articletype{Research Article}

\begin{document}

\title{Caracterización de sistemas de formación de imágenes}

\author{Adriana \ldots,\authormark{1} Alex G. Recuenco,\authormark{1} and Carlos España Castaño\authormark{1}}

\address{\authormark{1}Universidad Complutense de Madrid, Madrid, DC 28040, España}

\section{Introducción}
En esta práctica vamos a caracterizar un sistema de formación de imágenes mediante la determinación experimental de su función de transferencia de modulación (MTF). Para ello utilizaremos un método directo (el test de barras) y un método indirecto (el test de borde).


\section{TODO: SETUP, or whichever way you call it}


El sistema cuya MTF mediremos consiste en un objetivo de microscopio con apertura numérica $NA = 0,25$ y un aumento $10X$ (F10), y una cámara CCD con un tamaño de píxel de $s=4,65 \mu m$. Las imágenes captadas por la cámara se visualizarán en la pantalla del PC.

\subsection{Medida de MTF mediante el test de barras (método directo)}
\subsection{Medida de la MTF mediante el test de borde (método indirecto)}
\section{Resultados experimentales}

\subsection{Medida de la MTF mediante el test de barras (método directo)}

Utilizamos el test USAF 1951 (Fig.~\ref{fig:usaf1951}):

\begin{figure}[htbp]
	\centering
	\includegraphics[scale=0.05]{testusaf1951}
	\caption{Test USAF 1951}\label{fig:usaf1951}
\end{figure}

En la tabla~\ref{table:perfilintensidad} se recogen medidas tomadas del perfil de intensidad para distintos conjuntos de barras.

\begin{table}[p]
	\centering
	\csvautotabular[
		table head=\toprule%
		Barras & $y_{\min} (px)$ & $y_{\max} (px)$ & Contraste & $x_{\min} (px)$ & $x_{\max} (px)$ & N Periodos & periodo (px) &  $\flatfrac{\textrm{ciclos}}{\textrm{mm}}$%
		\\\midrule%
	]{practica1/MTF/Profiles.csv}
	\caption{Datos del perfil de intensidad. $y$: intensidad. $x$: distancia en píxeles. El contraste se ha obtenido a partir de la eq. \ref{eq:contraste}. la frequencia se ha obtenido a traves de la eq. \ref{eq:frecuencia}}%
	\label{table:perfilintensidad}
\end{table}

El contraste se ha obtenido a partir de la expresión
\nolinebreak
\begin{equation}
	C = \frac{y_{\max} - y_{\min}}{y_{\max} + y_{\min}}.
	\label{eq:contraste}
\end{equation}

La frecuencia, se calcula como:

\begin{equation}
	\nu = \frac{1825}{T}\ \textrm{ciclos/mm},\quad\textrm{TODO: QUE es 1825??}
	\label{eq:frecuencia}
\end{equation}

Donde el periodo, $T$, se obtiene dividiendo la diferencia $x_{\max} - x_{\min}$ entre el número de periodos en ese intervalo.

TODO: Incluir grafica de ejemplo, indicando como se obtinee

\subsection{Medida de la MTF mediante el test de borde (método indirecto)}

\section{Cuestiones}

\subsection{Cuestión 1}
Sea $H_{c}(u)$ la función de transferencia de un sistema para luz coherente. Escribir la relación entre $H_{c}(u)$ y la función de transferencia del mismo sistema para luz incoherente, $H_{i}(u)$.


\subsection{Cuestión 2}
Para un sistema parecido al usado en la práctica:
\begin{enumerate}
	\item Estimar la frecuencia de corte de la cámara CCD, $u_{c}^{(CCD)}$, en líneas por milímetro.

	\item Estimar la frecuencia de corte $u_{c}^{(Obj)}$ del objetivo $10\times$ para luz incoherente (la longitud de onda media $\lambda=500nm$).

	\item Tomando estos valores y suponiendo que la PSF del objetivo y de la cámara se aproximan por la función $|sinc(ax)|^2$, dibujar la MTF de cada uno de los elementos y del sistema compuesto.
\end{enumerate}


\subsection{Cuestión 3}
Aplicar usando el plugin Deconvolutionlab2 y Image J dos métodos de convolución: Regularized Inverse Filter y Richardson-Lucy a una imagen test obtenida por un sistema con una PSF conocida y diferentes tipos de ruido: Guass y Poisson.

%%%%%%%%%% If using BibTeX:
\bibliography{bibliography}

\end{document}
